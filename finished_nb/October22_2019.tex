\documentclass[]{article}
\usepackage{lmodern}
\usepackage{amssymb,amsmath}
\usepackage{ifxetex,ifluatex}
\usepackage{fixltx2e} % provides \textsubscript
\ifnum 0\ifxetex 1\fi\ifluatex 1\fi=0 % if pdftex
  \usepackage[T1]{fontenc}
  \usepackage[utf8]{inputenc}
\else % if luatex or xelatex
  \ifxetex
    \usepackage{mathspec}
  \else
    \usepackage{fontspec}
  \fi
  \defaultfontfeatures{Ligatures=TeX,Scale=MatchLowercase}
\fi
% use upquote if available, for straight quotes in verbatim environments
\IfFileExists{upquote.sty}{\usepackage{upquote}}{}
% use microtype if available
\IfFileExists{microtype.sty}{%
\usepackage{microtype}
\UseMicrotypeSet[protrusion]{basicmath} % disable protrusion for tt fonts
}{}
\usepackage[margin=1in]{geometry}
\usepackage{hyperref}
\hypersetup{unicode=true,
            pdftitle={Tidy\_Tuesday\_Oct22},
            pdfborder={0 0 0},
            breaklinks=true}
\urlstyle{same}  % don't use monospace font for urls
\usepackage{color}
\usepackage{fancyvrb}
\newcommand{\VerbBar}{|}
\newcommand{\VERB}{\Verb[commandchars=\\\{\}]}
\DefineVerbatimEnvironment{Highlighting}{Verbatim}{commandchars=\\\{\}}
% Add ',fontsize=\small' for more characters per line
\usepackage{framed}
\definecolor{shadecolor}{RGB}{248,248,248}
\newenvironment{Shaded}{\begin{snugshade}}{\end{snugshade}}
\newcommand{\KeywordTok}[1]{\textcolor[rgb]{0.13,0.29,0.53}{\textbf{#1}}}
\newcommand{\DataTypeTok}[1]{\textcolor[rgb]{0.13,0.29,0.53}{#1}}
\newcommand{\DecValTok}[1]{\textcolor[rgb]{0.00,0.00,0.81}{#1}}
\newcommand{\BaseNTok}[1]{\textcolor[rgb]{0.00,0.00,0.81}{#1}}
\newcommand{\FloatTok}[1]{\textcolor[rgb]{0.00,0.00,0.81}{#1}}
\newcommand{\ConstantTok}[1]{\textcolor[rgb]{0.00,0.00,0.00}{#1}}
\newcommand{\CharTok}[1]{\textcolor[rgb]{0.31,0.60,0.02}{#1}}
\newcommand{\SpecialCharTok}[1]{\textcolor[rgb]{0.00,0.00,0.00}{#1}}
\newcommand{\StringTok}[1]{\textcolor[rgb]{0.31,0.60,0.02}{#1}}
\newcommand{\VerbatimStringTok}[1]{\textcolor[rgb]{0.31,0.60,0.02}{#1}}
\newcommand{\SpecialStringTok}[1]{\textcolor[rgb]{0.31,0.60,0.02}{#1}}
\newcommand{\ImportTok}[1]{#1}
\newcommand{\CommentTok}[1]{\textcolor[rgb]{0.56,0.35,0.01}{\textit{#1}}}
\newcommand{\DocumentationTok}[1]{\textcolor[rgb]{0.56,0.35,0.01}{\textbf{\textit{#1}}}}
\newcommand{\AnnotationTok}[1]{\textcolor[rgb]{0.56,0.35,0.01}{\textbf{\textit{#1}}}}
\newcommand{\CommentVarTok}[1]{\textcolor[rgb]{0.56,0.35,0.01}{\textbf{\textit{#1}}}}
\newcommand{\OtherTok}[1]{\textcolor[rgb]{0.56,0.35,0.01}{#1}}
\newcommand{\FunctionTok}[1]{\textcolor[rgb]{0.00,0.00,0.00}{#1}}
\newcommand{\VariableTok}[1]{\textcolor[rgb]{0.00,0.00,0.00}{#1}}
\newcommand{\ControlFlowTok}[1]{\textcolor[rgb]{0.13,0.29,0.53}{\textbf{#1}}}
\newcommand{\OperatorTok}[1]{\textcolor[rgb]{0.81,0.36,0.00}{\textbf{#1}}}
\newcommand{\BuiltInTok}[1]{#1}
\newcommand{\ExtensionTok}[1]{#1}
\newcommand{\PreprocessorTok}[1]{\textcolor[rgb]{0.56,0.35,0.01}{\textit{#1}}}
\newcommand{\AttributeTok}[1]{\textcolor[rgb]{0.77,0.63,0.00}{#1}}
\newcommand{\RegionMarkerTok}[1]{#1}
\newcommand{\InformationTok}[1]{\textcolor[rgb]{0.56,0.35,0.01}{\textbf{\textit{#1}}}}
\newcommand{\WarningTok}[1]{\textcolor[rgb]{0.56,0.35,0.01}{\textbf{\textit{#1}}}}
\newcommand{\AlertTok}[1]{\textcolor[rgb]{0.94,0.16,0.16}{#1}}
\newcommand{\ErrorTok}[1]{\textcolor[rgb]{0.64,0.00,0.00}{\textbf{#1}}}
\newcommand{\NormalTok}[1]{#1}
\usepackage{graphicx,grffile}
\makeatletter
\def\maxwidth{\ifdim\Gin@nat@width>\linewidth\linewidth\else\Gin@nat@width\fi}
\def\maxheight{\ifdim\Gin@nat@height>\textheight\textheight\else\Gin@nat@height\fi}
\makeatother
% Scale images if necessary, so that they will not overflow the page
% margins by default, and it is still possible to overwrite the defaults
% using explicit options in \includegraphics[width, height, ...]{}
\setkeys{Gin}{width=\maxwidth,height=\maxheight,keepaspectratio}
\IfFileExists{parskip.sty}{%
\usepackage{parskip}
}{% else
\setlength{\parindent}{0pt}
\setlength{\parskip}{6pt plus 2pt minus 1pt}
}
\setlength{\emergencystretch}{3em}  % prevent overfull lines
\providecommand{\tightlist}{%
  \setlength{\itemsep}{0pt}\setlength{\parskip}{0pt}}
\setcounter{secnumdepth}{0}
% Redefines (sub)paragraphs to behave more like sections
\ifx\paragraph\undefined\else
\let\oldparagraph\paragraph
\renewcommand{\paragraph}[1]{\oldparagraph{#1}\mbox{}}
\fi
\ifx\subparagraph\undefined\else
\let\oldsubparagraph\subparagraph
\renewcommand{\subparagraph}[1]{\oldsubparagraph{#1}\mbox{}}
\fi

%%% Use protect on footnotes to avoid problems with footnotes in titles
\let\rmarkdownfootnote\footnote%
\def\footnote{\protect\rmarkdownfootnote}

%%% Change title format to be more compact
\usepackage{titling}

% Create subtitle command for use in maketitle
\providecommand{\subtitle}[1]{
  \posttitle{
    \begin{center}\large#1\end{center}
    }
}

\setlength{\droptitle}{-2em}

  \title{Tidy\_Tuesday\_Oct22}
    \pretitle{\vspace{\droptitle}\centering\huge}
  \posttitle{\par}
    \author{}
    \preauthor{}\postauthor{}
    \date{}
    \predate{}\postdate{}
  

\begin{document}
\maketitle

Set up working space

\begin{Shaded}
\begin{Highlighting}[]
\KeywordTok{rm}\NormalTok{(}\DataTypeTok{list =} \KeywordTok{ls}\NormalTok{())}
\end{Highlighting}
\end{Shaded}

\begin{Shaded}
\begin{Highlighting}[]
\KeywordTok{library}\NormalTok{(ggplot2)}
\KeywordTok{library}\NormalTok{(tidyr)}
\end{Highlighting}
\end{Shaded}

Get the data

\begin{Shaded}
\begin{Highlighting}[]
\NormalTok{horror_movies <-}\StringTok{ }\NormalTok{readr}\OperatorTok{::}\KeywordTok{read_csv}\NormalTok{(}\StringTok{"https://raw.githubusercontent.com/rfordatascience/tidytuesday/master/data/2019/2019-10-22/horror_movies.csv"}\NormalTok{)}
\end{Highlighting}
\end{Shaded}

\begin{verbatim}
## Parsed with column specification:
## cols(
##   title = col_character(),
##   genres = col_character(),
##   release_date = col_character(),
##   release_country = col_character(),
##   movie_rating = col_character(),
##   review_rating = col_double(),
##   movie_run_time = col_character(),
##   plot = col_character(),
##   cast = col_character(),
##   language = col_character(),
##   filming_locations = col_character(),
##   budget = col_character()
## )
\end{verbatim}

\begin{Shaded}
\begin{Highlighting}[]
\NormalTok{df<-horror_movies}
\end{Highlighting}
\end{Shaded}

\begin{Shaded}
\begin{Highlighting}[]
\KeywordTok{head}\NormalTok{(df)}
\end{Highlighting}
\end{Shaded}

\begin{verbatim}
## # A tibble: 6 x 12
##   title genres release_date release_country movie_rating review_rating
##   <chr> <chr>  <chr>        <chr>           <chr>                <dbl>
## 1 Gut ~ Drama~ 26-Oct-12    USA             <NA>                   3.9
## 2 The ~ Horror 13-Jan-17    USA             <NA>                  NA  
## 3 Slee~ Horror 21-Oct-17    Canada          <NA>                  NA  
## 4 Trea~ Comed~ 23-Apr-13    USA             NOT RATED              3.7
## 5 Infi~ Crime~ 10-Apr-15    USA             <NA>                   5.8
## 6 In E~ Horro~ 2017         UK              <NA>                  NA  
## # ... with 6 more variables: movie_run_time <chr>, plot <chr>, cast <chr>,
## #   language <chr>, filming_locations <chr>, budget <chr>
\end{verbatim}

Does review rating correpond to budget spent?

data wrangling

\begin{Shaded}
\begin{Highlighting}[]
\CommentTok{# subset data so only get values that we have}
\NormalTok{df.sub<-df[}\KeywordTok{complete.cases}\NormalTok{(df}\OperatorTok{$}\NormalTok{review_rating),]}
\NormalTok{df.sub<-df.sub[}\KeywordTok{complete.cases}\NormalTok{(df.sub}\OperatorTok{$}\NormalTok{budget),]}
\KeywordTok{head}\NormalTok{(df.sub)}
\end{Highlighting}
\end{Shaded}

\begin{verbatim}
## # A tibble: 6 x 12
##   title genres release_date release_country movie_rating review_rating
##   <chr> <chr>  <chr>        <chr>           <chr>                <dbl>
## 1 Rise~ Adven~ 1-May-12     USA             NOT RATED              3.6
## 2 Sexy~ Drama~ 21-Mar-17    USA             <NA>                   5.9
## 3 Circ~ Actio~ 13-Jan-17    USA             <NA>                   6  
## 4 Zomb~ Horror 23-Mar-15    UK              NOT RATED              2.7
## 5 Devi~ Horror 16-Sep-14    UK              <NA>                   3.4
## 6 Befo~ Horror 8-Jun-13     Japan           NOT RATED              4.7
## # ... with 6 more variables: movie_run_time <chr>, plot <chr>, cast <chr>,
## #   language <chr>, filming_locations <chr>, budget <chr>
\end{verbatim}

\begin{Shaded}
\begin{Highlighting}[]
\CommentTok{#budgets are in dolars, euros and pounds, lets only look at values in $}
\NormalTok{dfsub2<-df.sub[}\KeywordTok{grep}\NormalTok{(}\StringTok{"}\CharTok{\textbackslash{}\textbackslash{}}\StringTok{$"}\NormalTok{,df.sub}\OperatorTok{$}\NormalTok{budget),]}
\KeywordTok{head}\NormalTok{(dfsub2); }\KeywordTok{dim}\NormalTok{(dfsub2) }\CommentTok{# 847 movies}
\end{Highlighting}
\end{Shaded}

\begin{verbatim}
## # A tibble: 6 x 12
##   title genres release_date release_country movie_rating review_rating
##   <chr> <chr>  <chr>        <chr>           <chr>                <dbl>
## 1 Rise~ Adven~ 1-May-12     USA             NOT RATED              3.6
## 2 Circ~ Actio~ 13-Jan-17    USA             <NA>                   6  
## 3 Devi~ Horror 16-Sep-14    UK              <NA>                   3.4
## 4 Appa~ Fanta~ 5-May-15     USA             NOT RATED              4  
## 5 2: V~ Horror 1-Oct-12     USA             <NA>                   4.5
## 6 Her ~ Horror 19-Apr-13    USA             NOT RATED              5.4
## # ... with 6 more variables: movie_run_time <chr>, plot <chr>, cast <chr>,
## #   language <chr>, filming_locations <chr>, budget <chr>
\end{verbatim}

\begin{verbatim}
## [1] 847  12
\end{verbatim}

\begin{Shaded}
\begin{Highlighting}[]
\NormalTok{dfsub2}\OperatorTok{$}\NormalTok{dollars<-}\KeywordTok{gsub}\NormalTok{(}\StringTok{'}\CharTok{\textbackslash{}\textbackslash{}}\StringTok{$'}\NormalTok{,}\StringTok{''}\NormalTok{,dfsub2}\OperatorTok{$}\NormalTok{budget) }\CommentTok{# use reexpressions to reformat values }
\NormalTok{dfsub2}\OperatorTok{$}\NormalTok{dollars<-}\KeywordTok{as.numeric}\NormalTok{(}\KeywordTok{gsub}\NormalTok{(}\StringTok{','}\NormalTok{,}\StringTok{''}\NormalTok{,dfsub2}\OperatorTok{$}\NormalTok{dollars))}
\KeywordTok{range}\NormalTok{(dfsub2}\OperatorTok{$}\NormalTok{dollars)}
\end{Highlighting}
\end{Shaded}

\begin{verbatim}
## [1] 1.0e+02 1.9e+08
\end{verbatim}

\begin{Shaded}
\begin{Highlighting}[]
\CommentTok{#split filiming locations }
\KeywordTok{head}\NormalTok{(dfsub2)}
\end{Highlighting}
\end{Shaded}

\begin{verbatim}
## # A tibble: 6 x 13
##   title genres release_date release_country movie_rating review_rating
##   <chr> <chr>  <chr>        <chr>           <chr>                <dbl>
## 1 Rise~ Adven~ 1-May-12     USA             NOT RATED              3.6
## 2 Circ~ Actio~ 13-Jan-17    USA             <NA>                   6  
## 3 Devi~ Horror 16-Sep-14    UK              <NA>                   3.4
## 4 Appa~ Fanta~ 5-May-15     USA             NOT RATED              4  
## 5 2: V~ Horror 1-Oct-12     USA             <NA>                   4.5
## 6 Her ~ Horror 19-Apr-13    USA             NOT RATED              5.4
## # ... with 7 more variables: movie_run_time <chr>, plot <chr>, cast <chr>,
## #   language <chr>, filming_locations <chr>, budget <chr>, dollars <dbl>
\end{verbatim}

\section{Visualize results}\label{visualize-results}

\begin{Shaded}
\begin{Highlighting}[]
\KeywordTok{head}\NormalTok{(dfsub2)}
\end{Highlighting}
\end{Shaded}

\begin{verbatim}
## # A tibble: 6 x 13
##   title genres release_date release_country movie_rating review_rating
##   <chr> <chr>  <chr>        <chr>           <chr>                <dbl>
## 1 Rise~ Adven~ 1-May-12     USA             NOT RATED              3.6
## 2 Circ~ Actio~ 13-Jan-17    USA             <NA>                   6  
## 3 Devi~ Horror 16-Sep-14    UK              <NA>                   3.4
## 4 Appa~ Fanta~ 5-May-15     USA             NOT RATED              4  
## 5 2: V~ Horror 1-Oct-12     USA             <NA>                   4.5
## 6 Her ~ Horror 19-Apr-13    USA             NOT RATED              5.4
## # ... with 7 more variables: movie_run_time <chr>, plot <chr>, cast <chr>,
## #   language <chr>, filming_locations <chr>, budget <chr>, dollars <dbl>
\end{verbatim}

\begin{Shaded}
\begin{Highlighting}[]
\KeywordTok{ggplot}\NormalTok{(dfsub2, }\KeywordTok{aes}\NormalTok{(}\DataTypeTok{x=}\KeywordTok{log}\NormalTok{(dollars), }\DataTypeTok{y=}\NormalTok{review_rating)) }\OperatorTok{+}\StringTok{ }\KeywordTok{geom_point}\NormalTok{(}\DataTypeTok{alpha=}\FloatTok{0.75}\NormalTok{) }\OperatorTok{+}\StringTok{ }\KeywordTok{geom_smooth}\NormalTok{(}\DataTypeTok{method =} \StringTok{'auto'}\NormalTok{)}\OperatorTok{+}\StringTok{ }\KeywordTok{ylab}\NormalTok{(}\StringTok{'review rates'}\NormalTok{) }\OperatorTok{+}\StringTok{ }\KeywordTok{xlab}\NormalTok{(}\StringTok{'dollars'}\NormalTok{)}
\end{Highlighting}
\end{Shaded}

\begin{verbatim}
## `geom_smooth()` using method = 'loess' and formula 'y ~ x'
\end{verbatim}

\includegraphics{October22_2019_files/figure-latex/unnamed-chunk-8-1.pdf}

\begin{Shaded}
\begin{Highlighting}[]
\KeywordTok{ggsave}\NormalTok{(}\StringTok{'my_first_tidyTuesday.pdf'}\NormalTok{)}
\end{Highlighting}
\end{Shaded}

\begin{verbatim}
## Saving 6.5 x 4.5 in image
\end{verbatim}

\begin{verbatim}
## `geom_smooth()` using method = 'loess' and formula 'y ~ x'
\end{verbatim}


\end{document}
